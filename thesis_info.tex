\title{Título da Dissertação}{Subtítulo}
\alttitle{Title of the Dissertation}{Subtitle}
\author{Nome Completo do Autor}

\graduation{<nome do curso>}{<nome da especialização>}{<nome completo do mestrado, incluindo ramos/área se aplicável>}
\affiliation{<à nome da unidade orgânica>/<ao nome do departamento, se aplicável da/do nome da unidade orgânica>}

\supervisor{Professor Doutor}{Professor Auxiliar da Universidade de Coimbra}{<nome completo do orientador>}
\cosupervisor{Professor Doutor}{Professor Auxiliar da Universidade de Coimbra}{<nome completo do co-orientador>}
\president{Professor Doutor}{Professor Auxiliar da Universidade de Coimbra}{<nome completo do presidente do júri>}
\vowel{Professor Doutor}{Professor Auxiliar da Universidade de Coimbra}{<nome completo do vogal>}

\thesisdate{Mês}{20XX}

\institution{Nome da Instituição 1}{\includegraphics[width=3cm]{institutional/UC_V_Capa.eps}}
\coinstitution{Nome da Instituição 2}{\includegraphics[width=3cm]{institutional/UC_V_Capa.eps}}

\dedication{%
Aqui coloca-se a dedicatória
}

\acknowledgement{%
Aqui colocam-se os agradecimentos.
}

\thesisabstract{% Abstract in Main Language
Nesta parte irás escrever o resumo da tese. Deves apontar para 500 palavras com a seguinte estrutura:
\begin{itemize}
    \item \textbf{\emph{Lead Paragraph:}} 2-3 frases em que introduzes o leitor no contexto da tua tese. É o equivalente ao enquadramento da tese. O seu contexto.
    \item \textbf{\emph{Porquê:}} 2-3 frases em que procuras criar uma ligação com o leitor, apelando à importância daquilo que será estudado e "porquê".
    \item \textbf{\emph{Como:}} 3-4 frases em que explicas o essencial do método que usaste na abordagem ao tema. Entrar num ou noutro detalhe, mas sem exagero
    \item \textbf{\emph{O quê:}} 4-5 frases sobre os resultados obtidos e principais conclusões
    \item \textbf{\emph{Perspectiva:}} 1-2 frases sobre questões abertas, ou deixadas em aberto para o prosseguimento da investigação realizada \emph{realizada}.
\end{itemize}
}{% Abstract in Secondary Language
The abstract written in English should be similar to the abstract in Portuguese, but don't do a literal translation. It never works. The Portuguese language is rich in the passive voice, while in English we should use the active voice instead. An example.

"Three thermocouples \emph{were assembled} into the copper block and allow measuring its temperature." (Passive voice in italics)

"Three thermocouples \emph{assembled} into the copper block allow measuring its temperature." (Active voice). It's not easy, but there are several tools to help you in this task like Grammarly, ProWriting Aid and Hemingway. Use them.
}

\thesiskeywords{% Keywords in Main Language
    Palavra-chave 1, Palavra-chave 2, Palavra-chave 3, Palavra-chave 4
}{% Keywords in Secondary Language
    Keyword 1, Keyword 2, Keyword 3, Keyword 4
}